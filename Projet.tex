% Création : fév. 2015.
% Révision : Déc. 2015. pour appel 2016
\documentclass[10pt,reqno]{amsart}

\usepackage[utf8x]{inputenc}
\usepackage[francais]{babel}
\usepackage{amsmath,amsfonts,amssymb,amsthm,shuffle}
\usepackage[T1]{fontenc}
\usepackage{lmodern}

% Layout.
\usepackage[top=2cm,bottom=2cm,left=2.5cm,right=2.5cm]{geometry}

% Colors of hyperlinks.
\usepackage[dvipsnames]{xcolor}
\usepackage[hyperindex=true,frenchlinks=true,colorlinks=true,
citecolor=RoyalBlue,linkcolor=LimeGreen,urlcolor=Tan,linktocpage,
pagebackref=false]{hyperref}

% Tikz.
\usepackage{tikz}

% Misc.
\usepackage{cite}
\usepackage{subcaption}
\usepackage{ifthen}
\usepackage{textcomp} % symbole euro
\usepackage{calc} % encadrement de texte
\usepackage{array} % fixer la taille des colonnes
\usepackage{enumitem}

%%%%%%%%%%%%%%%%%%%%%%%%%%%%%%%%%%%%%%%%%%%%%%%%%%%%%%%%%%%%%%%%%%%%%%%%
%%%%%%%%%%%%%%%%%%%%%%%%%%%%%%%%%%%%%%%%%%%%%%%%%%%%%%%%%%%%%%%%%%%%%%%%
%%%%%%%%%%%%%%%%%%%%%%%%%%%%%%%%%%%%%%%%%%%%%%%%%%%%%%%%%%%%%%%%%%%%%%%%
\title{Projet PEPS \\
Dénombrement de prographes}
\date{\today}
\keywords{Combinatoire~; dénombrement~; opérades~; PROs~;
séries génératrices~; arbres.}

\author{Nicolas Borie}
\author{Jean-Paul Bultel}
\author{Samuele Giraudo}
\author{Matthieu Josuat-Vergès}

%%%%%%%%%%%%%%%%%%%%%%%%%%%%%%%%%%%%%%%%%%%%%%%%%%%%%%%%%%%%%%%%%%%%%%%%
%%%%%%%%%%%%%%%%%%%%%%%%%%%%%%%%%%%%%%%%%%%%%%%%%%%%%%%%%%%%%%%%%%%%%%%%
%%%%%%%%%%%%%%%%%%%%%%%%%%%%%%%%%%%%%%%%%%%%%%%%%%%%%%%%%%%%%%%%%%%%%%%%
\numberwithin{equation}{subsection}
\setcounter{tocdepth}{2}
\renewcommand{\leq}{\leqslant}
\renewcommand{\geq}{\geqslant}

%%%%%%%%%%%%%%%%%%%%%%%%%%%%%%%%%%%%%%%%%%%%%%%%%%%%%%%%%%%%%%%%%%%%%%%%
%%%%%%%%%%%%%%%%%%%%%%%%%%%%%%%%%%%%%%%%%%%%%%%%%%%%%%%%%%%%%%%%%%%%%%%%
%%%%%%%%%%%%%%%%%%%%%%%%%%%%%%%%%%%%%%%%%%%%%%%%%%%%%%%%%%%%%%%%%%%%%%%%
\tikzstyle{Noeud} = [circle,draw=Turquoise!80,fill=Turquoise!20,thick,
inner sep=0pt,minimum size=4mm,line width=1pt]
\tikzstyle{Arete}=[OrangeRed!80,cap=round,line width=1pt]
\tikzstyle{Feuille}=[rectangle,draw=Black!70,fill=Black!20,
inner sep=0pt,minimum size=1mm,line width=.75pt]
\tikzstyle{Operateur}=[rectangle,rounded corners,draw=YellowGreen!80,
fill=YellowGreen!20,minimum size=5mm,inner sep=1pt,line width=1pt]
\tikzstyle{Marque1}=[draw=YellowOrange!80,fill=YellowOrange!20]
\tikzstyle{Marque2}=[draw=SeaGreen!80,fill=SeaGreen!20]
\tikzstyle{Marque3}=[draw=Maroon!80,fill=Maroon!20]
\tikzstyle{Marque4}=[draw=Periwinkle!80,fill=Periwinkle!20]

%%%%%%%%%%%%%%%%%%%%%%%%%%%%%%%%%%%%%%%%%%%%%%%%%%%%%%%%%%%%%%%%%%%%%%%%
%%%%%%%%%%%%%%%%%%%%%%%%%%%%%%%%%%%%%%%%%%%%%%%%%%%%%%%%%%%%%%%%%%%%%%%%
%%%%%%%%%%%%%%%%%%%%%%%%%%%%%%%%%%%%%%%%%%%%%%%%%%%%%%%%%%%%%%%%%%%%%%%%
\newcommand{\Hilbert}{\mathcal{H}}
\newcommand{\Gen}{\mathcal{G}}
\newcommand{\OpLibre}{\mathrm{OL}}
\newcommand{\PROLibre}{\mathrm{PL}}
\newcommand{\Entrees}{\mathrm{ent}}
\newcommand{\Sorties}{\mathrm{sor}}
\newcommand{\Degre}{\mathrm{deg}}

\usepackage{ifthen}
\newboolean{draft}
\setboolean{draft}{true} % true affiche les todo et info, false les degagent
\ifdraft
\newcommand{\TODO}[2][To do: ]{\textcolor{red}{\textit{#1#2}}}
\newcommand{\INFO}[2][Info: ]{\textcolor{green}{\textit{#1#2}}}
\else
\newcommand{\TODO}[2][]{}
\newcommand{\INFO}[2][]{}
\fi

\newcommand{\Cacher}[1]{} % Commande qui efface son argument !

%%%%%%%%%%%%%%%%%%%%%%%%%%%%%%%%%%%%%%%%%%%%%%%%%%%%%%%%%%%%%%%%%%%%%%%%
%%%%%%%%%%%%%%%%%%%%%%%%%%%%%%%%%%%%%%%%%%%%%%%%%%%%%%%%%%%%%%%%%%%%%%%%
%%%%%%%%%%%%%%%%%%%%%%%%%%%%%%%%%%%%%%%%%%%%%%%%%%%%%%%%%%%%%%%%%%%%%%%%
\begin{document}

\begin{center}
\begin{tabular}{|>{\centering\arraybackslash}m{5.5cm}|>
    {\centering\arraybackslash}m{9.6cm}|} \hline
    \includegraphics{logo_CNRS.jpg} & \Large{APPEL A PROJETS} \\
    INS2I - 2016 & \Large{JCJC INS2I 2016} \\ \hline
\end{tabular}
\end{center}
\bigskip

\Large{\bf{Identification}}

\begin{center}
\begin{small}
\begin{tabular}{|m{5.5cm}|m{9.6cm}|} \hline
    Nom du porteur du projet & Samuele Giraudo (M.C.F. LIGM - UMR 8049)
        \\ \hline
    Adresse e-mail du porteur & {\tt samuele.giraudo@u-pem.fr} \\ \hline
    Titre long du projet & Pr{\bf o}priétés {\bf c}ombinatoires et
    {\bf a}lgébriques des {\bf P}ROs l{\bf i}bres\\ \hline
    Acronyme du projet & OCAPI \\ \hline
\end{tabular}
\end{small}
\end{center}
\bigskip

\Large{\bf{Résumé du projet}}

% ATTENTION : 10 lignes au maximum sont autorisées !
\fbox{\parbox{\linewidth-8\fboxrule-8\fboxsep}{Les prographes
    sont des assemblages plans de composants ayant plusieurs entrées et
    sorties. Suivant la nature de ses composants, un prographe permet de 
    modéliser une expression en calcul symbolique (les composants sont 
    des opérateurs algébriques), un graphe (les composants sont des 
    sommets) ou encore un circuit électronique. Ces objets combinatoires
    généralisent un bon nombre de structures discrètes comme les arbres 
    où chaque composant dispose d'une unique sortie, les permutations, 
    les cartes et les opérades mais sont encore des structures peu 
    étudiées. Par exemple, la conception d'un algorithme efficace pour 
    leur génération exhaustive est déjà un défi. D'un point de vue 
    algébrique, les prographes sont les éléments d'une structure connue 
    sous le nom de PRO libre. Ce projet propose une étude à la fois 
    combinatoire et algébrique des prographes.
}}
\bigskip

\Large{\bf{Autres membres du projet}}

\begin{center}
\begin{small}
\begin{tabular}{|l|c|l|c|}\hline
    \bf{Nom partenaire} & \bf{Qualité/} & \bf{e-mail partenaire}
        & \bf{Unité de} \\
    & \bf{Titre} &  & \bf{recherche} \\ \hline
    Nicolas Borie & M.C.F. & {\tt nicolas.borie@u-pem.fr}
        & LIGM - UMR 8049 \\ \hline
    Jean-Paul Bultel & Docteur & {\tt jean-paul.bultel@univ-rouen.fr}
        & LITIS - EA 4108 \\ \hline
  Matthieu Josuat-Vergès \qquad & C.R.2 & {\tt matthieu.josuat-verges@u-pem.fr}
   \qquad
    & LIGM - UMR 8049 \qquad \\ \hline
\end{tabular}
\end{small}
\end{center}
\bigskip

\Large{\bf{Description du projet}}

%%%%%%%%%%%%%%%%%%%%%%%%%%%%%%%%%%%%%%%%%%%%%%%%%%%%%%%%%%%%%%%%%%%%%%%%
%%%%%%%%%%%%%%%%%%%%%%%%%%%%%%%%%%%%%%%%%%%%%%%%%%%%%%%%%%%%%%%%%%%%%%%%
%%%%%%%%%%%%%%%%%%%%%%%%%%%%%%%%%%%%%%%%%%%%%%%%%%%%%%%%%%%%%%%%%%%%%%%%
\section{Contexte}

%%%%%%%%%%%%%%%%%%%%%%%%%%%%%%%%%%%%%%%%%%%%%%%%%%%%%%%%%%%%%%%%%%%%%%%%
%%%%%%%%%%%%%%%%%%%%%%%%%%%%%%%%%%%%%%%%%%%%%%%%%%%%%%%%%%%%%%%%%%%%%%%%
\subsection{Arbres syntaxiques et opérades}
Les arbres syntaxiques sont des structures arborescentes qui apparaissent 
dans de multiples contextes. Ils figurent en effet à l'intersection de
l'informatique et des mathématiques car ils interviennent, par exemple,
à la fois en compilation et en théorie des langages, et également en 
algèbre universelle et en théorie des opérades.

Leur principal intérêt est d'offrir une représentation d'expressions 
mélangeant valeurs et opérateurs, et de proposer un cadre cohérent pour 
réaliser des calculs. De plus, en s'autorisant la manipulation de 
paramètres, ils permettent de modéliser des relations vérifiées par 
des opérations. Par exemple, la relation d'associativité 
$(x \cdot y) \cdot z = x \cdot (y \cdot z)$ d'un opérateur binaire 
$\cdot$ se traduit en l'identification d'arbres syntaxiques
\begin{equation}
    \begin{split}
    \begin{tikzpicture}[xscale=.24,yscale=.15]
        \node[Feuille](S1)at(0,1){};
        \node[Feuille](E1)at(-4,-9){};
        \node[Feuille](E2)at(-2,-9){};
        \node[Feuille](E3)at(2,-9){};
        \node[Operateur](N1)at(0,-2){\begin{math}\cdot\end{math}};
        \node[Operateur](N2)at(-3,-6){\begin{math}\cdot\end{math}};
        \draw[Arete](N1)--(N2);
        \draw[Arete](S1)--(N1);
        \draw[Arete](E1)--(N2);
        \draw[Arete](E2)--(N2);
        \draw[Arete](E3)--(N1);
    \end{tikzpicture}
    \end{split}
    \enspace = \enspace
    \begin{split}
    \begin{tikzpicture}[xscale=.24,yscale=.15]
        \node[Feuille](S1)at(0,1){};
        \node[Feuille](E1)at(-2,-9){};
        \node[Feuille](E2)at(2,-9){};
        \node[Feuille](E3)at(4,-9){};
        \node[Operateur](N1)at(0,-2){\begin{math}\cdot\end{math}};
        \node[Operateur](N2)at(3,-6){\begin{math}\cdot\end{math}};
        \draw[Arete](N1)--(N2);
        \draw[Arete](S1)--(N1);
        \draw[Arete](E1)--(N1);
        \draw[Arete](E2)--(N2);
        \draw[Arete](E3)--(N2);
    \end{tikzpicture}
    \end{split}\,.
\end{equation}

Les opérades~\cite{Mar08,LV12} sont des structures offrant une vision 
algébrique des arbres syntaxiques. Les éléments d'une opérade sont en 
effet des opérateurs et dans une opérade, il est possible de composer 
deux opérateurs pour en construire de plus gros. Au même titre que les 
monoïdes libres peuvent être étudiés par le biais de la combinatoire des 
mots, les opérades libres peuvent l'être par la combinatoire des arbres.

Outre le fait que la théorie des opérades provient d'un contexte purement
algébrique, elle occupe aujourd'hui une place de choix en combinatoire. 
En effet, la définition d'opérades sur des objets combinatoires permet 
d'en extraire des propriétés, d'obtenir des résultats énumératifs ou 
encore d'établir des bijections entre différentes familles d'objets
combinatoires~\cite{Cha08,CG14,Gir15}. De plus, l'étude des opérades 
fait appel à des outils combinatoires classiques, comme les séries 
génératrices et la combinatoire de diverses familles d'arbres.

%%%%%%%%%%%%%%%%%%%%%%%%%%%%%%%%%%%%%%%%%%%%%%%%%%%%%%%%%%%%%%%%%%%%%%%%
%%%%%%%%%%%%%%%%%%%%%%%%%%%%%%%%%%%%%%%%%%%%%%%%%%%%%%%%%%%%%%%%%%%%%%%%
\subsection{Arbres syntaxiques généralisés et PROs}
Les arbres syntaxiques classiques --- et de ce fait leurs pendants 
algébriques que sont les opérades --- font preuve d'une limitation 
essentielle. Ils permettent seulement de modéliser des opérations et des
expressions ayant exactement une sortie. Des expressions dans des 
algèbres de Hopf, qui disposent d'un coproduit, une opération à une 
entrée et deux sorties, ne peuvent ainsi pas être modélisées par ces 
arbres syntaxiques. Il est ainsi naturel de considérer des structures
syntaxiques qui ne sont plus des arbres mais plutôt des graphes orientés, 
où les n\oe uds internes disposent potentiellement de plusieurs entrées 
et de plusieurs sorties.

À titre d'exemple, la relation axiomatique entre le produit $\cdot$ et 
le coproduit $\Delta$ d'une algèbre de Hopf s'écrit 
$\Delta(x \cdot y) = \Delta(x) \cdot \Delta(y)$ et celle-ci se traduit 
en l'identification d'arbres syntaxiques généralisés
\begin{equation}
    \begin{split}
    \begin{tikzpicture}[xscale=.24,yscale=.19]
        \node[Feuille](S1)at(0,1){};
        \node[Feuille](S2)at(2,1){};
        \node[Feuille](E1)at(0,-9){};
        \node[Feuille](E2)at(2,-9){};
        \node[Operateur,Marque3](N1)at(1,-2){\begin{math}\Delta\end{math}};
        \node[Operateur](N2)at(1,-6){\begin{math}\cdot\end{math}};
        \draw[Arete](N1)--(N2);
        \draw[Arete](E1)--(N2);
        \draw[Arete](E2)--(N2);
        \draw[Arete](S1)--(N1);
        \draw[Arete](S2)--(N1);
    \end{tikzpicture}
    \end{split}
    \enspace = \enspace
    \begin{split}
    \begin{tikzpicture}[xscale=.24,yscale=.19]
        \node[Feuille](S1)at(-2,1){};
        \node[Feuille](S2)at(2,1){};
        \node[Feuille](E1)at(-2,-9){};
        \node[Feuille](E2)at(2,-9){};
        \node[Operateur](N1)at(-2,-2){\begin{math}\cdot\end{math}};
        \node[Operateur](N2)at(2,-2){\begin{math}\cdot\end{math}};
        \node[Operateur,Marque3](N3)at(-2,-6){\begin{math}\Delta\end{math}};
        \node[Operateur,Marque3](N4)at(2,-6){\begin{math}\Delta\end{math}};
        \draw[Arete](N1)--(N4);
        \draw[Arete](N2)--(N3);
        \draw[Arete](N1)--(N3);
        \draw[Arete](N2)--(N4);
        \draw[Arete](E1)--(N3);
        \draw[Arete](E2)--(N4);
        \draw[Arete](S1)--(N1);
        \draw[Arete](S2)--(N2);
    \end{tikzpicture}
    \end{split}\,.
\end{equation}

Les PROs ({\em product category}) sont des structures algébriques 
introduites par Mac Lane~\cite{McL65} (de manière surprenante 
historiquement avant les opérades) qui généralisent les opérades et 
permettent de décrire les algèbres de Hopf ainsi que d'autres structures
algébriques encore plus complexes. Un PRO est ainsi un ensemble 
d'opérateurs à un nombre arbitraire d'entrées et de sorties, que l'on 
peut composer entre eux.

Les PROs libres~\cite{Laf03,Mar08,BG14} sont aux arbres syntaxiques 
généralisés ce que les opérades libres sont aux arbres syntaxiques. Dans 
ce contexte, un arbre syntaxique généralisé est appelé {\em prographe}. 
La figure~\ref{fig:exemple_prographe} donne un exemple d'un tel objet.
\begin{figure}[ht]
    \centering
    \begin{equation*}\footnotesize
        \begin{split}
        \begin{tikzpicture}[xscale=.24,yscale=.21]
            \node[Feuille](S1)at(2,0){};
            \node[Feuille](S2)at(4,0){};
            \node[Feuille](S3)at(5,0){};
            \node[Feuille](S4)at(7,0){};
            \node[Feuille](S5)at(9,0){};
            \node[Feuille](S6)at(13,0){};
            \node[Feuille](S7)at(16,0){};
            \node[Feuille](S8)at(18,0){};
            \node[Feuille](S9)at(19,0){};
            %
            \node[Feuille](E1)at(2,-10){};
            \node[Feuille](E2)at(4,-10){};
            \node[Feuille](E3)at(5,-10){};
            \node[Feuille](E4)at(6,-10){};
            \node[Feuille](E5)at(8,-10){};
            \node[Feuille](E6)at(10,-10){};
            \node[Feuille](E7)at(12,-10){};
            \node[Feuille](E8)at(14,-10){};
            \node[Feuille](E9)at(15,-10){};
            \node[Feuille](EA)at(16,-10){};
            \node[Feuille](EB)at(17,-10){};
            \node[Feuille](EC)at(18,-10){};
            \node[Feuille](ED)at(19,-10){};
            %
            \node[Operateur](N1)at(3,-5){\begin{math}{\tt a}\end{math}};
            \node[Operateur](N2)at(7,-8){\begin{math}{\tt a}\end{math}};
            \node[Operateur,Marque3](N3)at(9,-2){\begin{math}{\tt b}\end{math}};
            \node[Operateur](N4)at(11,-8){\begin{math}{\tt a}\end{math}};
            \node[Operateur](N5)at(13,-5){\begin{math}{\tt a}\end{math}};
            \node[Operateur,Marque2](N6)at(16,-5){\begin{math}{\tt c}\end{math}};
            %
            \draw[Arete](N1)--(S1);
            \draw[Arete](N1)--(S2);
            \draw[Arete](E3)--(S3);
            \draw[Arete](N2)--(S4);
            \draw[Arete](N3)--(S5);
            \draw[Arete](N5)--(S6);
            \draw[Arete](N6)--(S7);
            \draw[Arete](EC)--(S8);
            \draw[Arete](ED)--(S9);
            %
            \draw[Arete](N1)--(E1);
            \draw[Arete](N1)--(E2);
            \draw[Arete](N2)--(E4);
            \draw[Arete](N2)--(E5);
            \draw[Arete](N4)--(E6);
            \draw[Arete](N4)--(E7);
            \draw[Arete](N5)--(E8);
            \draw[Arete](N6)--(E9);
            \draw[Arete](N6)--(EA);
            \draw[Arete](N6)--(EB);
            %
            \draw[Arete](N2)--(N3);
            \draw[Arete](N3)--(N4);
            \draw[Arete](N3)--(N5);
            \draw[Arete](N4)--(N5);
        \end{tikzpicture}
        \end{split}
    \end{equation*}
    \caption{Un prographe sur des opérateurs ${\tt a}$, ${\tt b}$ 
    et ${\tt c}$ ayant respectivement deux entrées et deux sorties, 
    deux entrées et une sortie, et trois entrées et une sortie.
    Ce prographe code lui-même une opération ayant treize entrées
    et neuf sorties.}
    \label{fig:exemple_prographe}
\end{figure}

%%%%%%%%%%%%%%%%%%%%%%%%%%%%%%%%%%%%%%%%%%%%%%%%%%%%%%%%%%%%%%%%%%%%%%%%
%%%%%%%%%%%%%%%%%%%%%%%%%%%%%%%%%%%%%%%%%%%%%%%%%%%%%%%%%%%%%%%%%%%%%%%%
%%%%%%%%%%%%%%%%%%%%%%%%%%%%%%%%%%%%%%%%%%%%%%%%%%%%%%%%%%%%%%%%%%%%%%%%
\section{Questions et objectifs}
Toute la recherche prévue par ce projet porte sur la compréhension des 
prographes. Nous souhaitons utiliser ces objets à la fois pour modéliser 
des phénomènes déjà explorés et connus et pour se poser de nouvelles 
questions. Voici les directions principales~:
\begin{enumerate}[fullwidth,label={\bf (\arabic*)}]
    \item \label{item:objectif_denombrement}
    étant donné un ensemble d'opérateurs disposant de plusieurs entrées 
    et de plusieurs sorties, nous souhaitons dénombrer les prographes 
    sur ces opérateurs en fonction du nombre d'opérateurs utilisés. Cette
    question semble pouvoir se traiter d'un point de vue combinatoire en 
    utilisant et/ou généralisant des résultats sur les empilements de 
    pièces de Viennot~\cite{Vie86}~;
    
    \item \label{item:objectif_modelisation}
    nous souhaitons mesurer l'expressivité des prographes au sens 
    suivant. Les mots et les arbres sont des structures de données qui 
    permettent de représenter un bon nombre d'objets, les seconds étant
    une généralisation des premiers, leur expressivité est d'autant plus
    forte. Comme les prographes généralisent les arbres, nous pouvons
    nous attendre à pouvoir exprimer, par des prographes, de nouveaux
    objets et phénomènes. Cet axe de recherche consiste à délimiter cette 
    zone d'expressivité. À titre d'exemple, un prographe permet de 
    modéliser de manière évidente des circuits électroniques (voir 
    la figure~\ref{fig:exemple_prographe_circuit})
    \begin{figure}[ht]
        \centering
        \begin{tikzpicture}[yscale=0.5, xscale=0.7]
            \tikzstyle{Boite} = [very thick]
            \tikzstyle{Profils} = [thick]
            %
            % Cadre autour du PROgraphe
            \draw (0, 1) -- (5, 1);
            \draw (0, 1) -- (0, 8);
            \draw (5, 1) -- (5, 8);
            \draw (0, 8) -- (5, 8);
            %
            % Opérateurs boites du PROgraphe
            \draw[Boite] (3.5, 2)--(3.5, 3)--(4.5, 3)--(4.5, 2)--(3.5, 2);
            \draw[Boite] (0.5, 2)--(0.5, 3)--(2.5, 3)--(2.5, 2)--(0.5, 2);
            \draw[Boite] (0.5, 4)--(0.5, 5)--(3.5, 5)--(3.5, 4)--(0.5, 4);
            \draw[Boite] (2.5, 6)--(2.5, 7)--(4.5, 7)--(4.5, 6)--(2.5, 6);
            \draw[Boite] (0.5, 6)--(0.5, 7)--(1.5, 7)--(1.5, 6)--(0.5, 6);
            %
            % Points de soudure des boites
            \fill[black] (1, 2) circle (0.065cm);
            \fill[black] (2, 2) circle (0.065cm);
            \fill[black] (1.5, 3) circle (0.065cm);
            \fill[black] (4, 3) circle (0.065cm);
            \fill[black] (2.5, 4) circle (0.065cm);
            \fill[black] (4, 2) circle (0.065cm);
            \fill[black] (1.5, 4) circle (0.065cm);
            \fill[black] (1.25, 5) circle (0.065cm);
            \fill[black] (2, 5) circle (0.065cm);
            \fill[black] (2.75, 5) circle (0.065cm);
            \fill[black] (4, 6) circle (0.065cm);
            \fill[black] (1, 6) circle (0.065cm);
            \fill[black] (3, 6) circle (0.065cm);
            \fill[black] (1, 7) circle (0.065cm);
            \fill[black] (3.5, 7) circle (0.065cm);
            %
            % Fils du PROgraphe
            \draw[Profils] (1, 0.5) -- (1, 2);
            \draw[Profils] (2, 0.5) -- (2, 2);
            \draw[Profils] (3, 0.5) -- (3, 3.5) -- (2.5, 3.5) -- (2.5, 4);
            \draw[Profils] (4, 0.5) -- (4, 2);
            \draw[Profils] (1.5, 3) -- (1.5, 4);
            \draw[Profils] (4, 3) -- (4, 6);
            \draw[Profils] (1.25, 5) -- (1.25, 5.5) -- (1, 5.5) -- (1, 6);
            \draw[Profils] (2.75, 5) -- (2.75, 5.5) -- (3, 5.5) -- (3, 6);
            \draw[Profils] (1, 7) -- (1, 7.5) -- (1.25, 7.5) -- (1.25, 8.5);
            \draw[Profils] (2, 5) -- (2, 7.5) -- (2.5, 7.5) -- (2.5, 8.5);
            \draw[Profils] (3.5, 7) -- (3.5, 7.5) -- (3.75, 7.5) -- (3.75, 8.5);
            %
            % Entrées du PROgraphe
            \fill[black] (1, 0.5) circle (0.15cm);
            \draw (1, 0.5) node[below] {$e_1$};
            \fill[black] (2, 0.5) circle (0.15cm);
            \draw (2, 0.5) node[below] {$e_2$};
            \fill[black] (3, 0.5) circle (0.15cm);
            \draw (3, 0.5) node[below] {$e_3$};
            \fill[black] (4, 0.5) circle (0.15cm);
            \draw (4, 0.5) node[below] {$e_4$};
            %
            % Sorties du PROgraphe
            \fill[black] (1.25, 8.5) circle (0.15cm);
            \draw (1.25, 8.5) node[above] {$s_1$};
            \fill[black] (2.5, 8.5) circle (0.15cm);
            \draw (2.5, 8.5) node[above] {$s_2$};
            \fill[black] (3.75, 8.5) circle (0.15cm);
            \draw (3.75, 8.5) node[above] {$s_3$};
        \end{tikzpicture}
    \caption{Un prographe constitué de quatre entrées $e_1$, $e_2$, 
    $e_3$ et $e_4$ et de trois sorties $s_1$, $s_2$ et $s_3$. Il 
    contient deux composants ayant une entrée et une sortie, deux 
    composants ayant deux entrées et une sortie et un composant ayant 
    deux entrées et trois sorties.}
    \label{fig:exemple_prographe_circuit}
    \end{figure}
    et nous souhaitons savoir ce que cette vision combinatoire et 
    algébrique des circuits peut apporter~;
    
    \item \label{item:objectif_groupes}
    les PROs permettent de travailler avec {\em tous} les groupes 
    symétriques à la fois, au sens où il existe un PRO 
    (voir~\cite{Laf03,Laf13}) qui modélise les permutations (de toutes 
    tailles) et leurs compositions. En un certain sens, ce PRO renferme
    toute la combinatoire de tous  les groupes symétriques. La question 
    de savoir s'il existe d'autres groupes subissant un tel traitement
    (comme les groupes de Coxeter ou les groupes de tresses) et de 
    savoir ce que cette vision peut apporter nous intéresse~;
    
    \item \label{item:objectif_cartes}
    les prographes peuvent être décrits comme des cartes soumises à 
    certaines conditions. À l'inverse, il est possible de coder certaines
    cartes par des prographes, eux-même soumis à certaines restrictions.
    Nous espérons dans ce projet créer une interaction riche et nouvelle 
    entre la théorie des cartes et la théorie des PROs. Il semble d'une 
    part pouvoir être possible d'utiliser certains résultats énumératifs 
    des cartes pour le dénombrement des prographes et certains résultats 
    algébriques des PROs en théorie des cartes.
\end{enumerate}

Ces axes de recherche étant variés, nous espérons à l'issue de 
l'exploration de quelques-unes de ces questions, soumettre un article à 
un journal de combinatoire algébrique ainsi qu'un article à une 
conférence du même domaine. Si les résultats et la direction prise
par notre recherche nous le permet, il nous sera envisageable de soumettre
certains résultats à un journal d'informatique théorique.

%%%%%%%%%%%%%%%%%%%%%%%%%%%%%%%%%%%%%%%%%%%%%%%%%%%%%%%%%%%%%%%%%%%%%%%%
%%%%%%%%%%%%%%%%%%%%%%%%%%%%%%%%%%%%%%%%%%%%%%%%%%%%%%%%%%%%%%%%%%%%%%%%
%%%%%%%%%%%%%%%%%%%%%%%%%%%%%%%%%%%%%%%%%%%%%%%%%%%%%%%%%%%%%%%%%%%%%%%%
\section{Résultats partiels et observations}
Dans un travail récent~\cite{BG14}, certains des membres de ce projet 
ont établi des résultats partiels pour 
l'axe~\eqref{item:objectif_denombrement} lorsque un unique opérateur 
ayant le même nombre d'entrées et de sorties est autorisé. Ce résultat 
s'appuie sur le {\em lemme d'inversion} de Viennot~\cite{Vie86},
introduit dans le contexte du dénombrement d'empilements de pièces. 

Listons à présent quelques résultats expérimentaux.

\TODO{Insérer ici les résultats et observations de Nicolas.}

\TODO{Compléter.}

%%%%%%%%%%%%%%%%%%%%%%%%%%%%%%%%%%%%%%%%%%%%%%%%%%%%%%%%%%%%%%%%%%%%%%%%
%%%%%%%%%%%%%%%%%%%%%%%%%%%%%%%%%%%%%%%%%%%%%%%%%%%%%%%%%%%%%%%%%%%%%%%%
%%%%%%%%%%%%%%%%%%%%%%%%%%%%%%%%%%%%%%%%%%%%%%%%%%%%%%%%%%%%%%%%%%%%%%%%
\section{Membres du projet}
Nous décrivons ici la constitution de l'équipe du projet et expliquons
en quoi elle nous semble adaptée. Le domaine de recherche des quatre 
membres s'inscrit en combinatoire algébrique~; chacun possède néanmoins 
ses spécialités.

Le projet est porté par {\bf Samuele Giraudo}
(\url{http://igm.univ-mlv.fr/~giraudo/}). Il a effectué sa thèse à
l'université Paris-Est Marne-la-Vallée au LIGM sous la direction de
Jean-Christophe Novelli et Florent Hivert, soutenue le 8 décembre 2011.
Il occupe depuis 2012 un poste de maître de conférences en informatique
au LIGM dans l'équipe combinatoire algébrique et calcul symbolique. Son
domaine principal est la théorie des opérades. Il a présenté
une construction générale d'opérades à partir de monoïdes~\cite{Gir15} 
et un moyen de comprendre les propriétés d'une opérade en la voyant 
comme une certaine opérade colorée~\cite{CG14}. Ce membre a aussi obtenu 
des résultats sur les PROs~\cite{BG14}. Étant donné que la plupart des 
notions connues sur les opérades peuvent se généraliser aux PROs, ses
connaissances sur les opérades pourront être utiles dans ce travail.

{\bf Nicolas Borie} (\url{http://www-igm.univ-mlv.fr/~borie/}) a effectué
sa thèse à l'université Paris-Sud XI au LRI sous la direction de Nicolas
Thiéry, soutenue le 7 décembre 2011. Il occupe depuis 2013 un poste de
maître de conférences en informatique au LIGM dans l'équipe combinatoire
algébrique et calcul symbolique. Ce membre développe depuis de nombreuses
années pour \texttt{Sage}. Ses connaissances en programmation et en
expérimentation informatique seront nécessaires pour mener le projet à
bien. L'un de ses résultats~\cite{Bor13} consiste en la conception d'un
algorithme efficace de génération de mots d'entiers soumis à certaines
conditions, dont certaines idées semblent pouvoir être exploitées pour
la génération des prographes. Son domaine principal est la théorie
effective des invariants, domaine qu'il est le seul à avoir exploré
parmi les membres de ce groupe.

{\bf Jean-Paul Bultel} a effectué sa thèse à l'université Paris-Est
Marne-la-Vallée au LIGM sous la direction de Jean-Yves Thibon, soutenue
le 25 novembre 2011. Il est actuellement membre associé au LITIS 
(Université de Rouen). Ce membre a déjà travaillé et obtenu des
résultats sur les PROs~\cite{BG14}. Son domaine principal est la théorie
des algèbres de Hopf combinatoires (voir~\cite{Bul11}). Ses connaissances
des structures algébriques sur des opérateurs à plusieurs entrées et
plusieurs sorties sera nécessaire au déroulement de ce travail.

{\bf Matthieu Josuat-Vergès} (\url{http://igm.univ-mlv.fr/~josuatv/})
a effectué sa thèse à l'université Paris-Sud XI au LRI sous la
direction de Sylvie Corteel, soutenue le 25 janvier 2010. Il occupe
depuis 2011 un poste de chargé de recherche CNRS au LIGM dans l'équipe
combinatoire algébrique et calcul symbolique. Ce membre possède une
vision très large et l'expérience la plus grande en combinatoire parmi
les membres de ce groupe. Il a par ailleurs récemment obtenu des
résultats énumératifs sur des familles d'arbres~\cite{JV15}. Ses
connaissances en combinatoire vont potentiellement pouvoir mettre en
évidence, au fil du traitement de ce projet, des connexions non
évidentes entre le sujet du projet et d'autres thèmes.

%%%%%%%%%%%%%%%%%%%%%%%%%%%%%%%%%%%%%%%%%%%%%%%%%%%%%%%%%%%%%%%%%%%%%%%%
%%%%%%%%%%%%%%%%%%%%%%%%%%%%%%%%%%%%%%%%%%%%%%%%%%%%%%%%%%%%%%%%%%%%%%%%
%%%%%%%%%%%%%%%%%%%%%%%%%%%%%%%%%%%%%%%%%%%%%%%%%%%%%%%%%%%%%%%%%%%%%%%%
\section{Budget}
Le budget demandé est 14000\texteuro. Voici sa répartition.
\smallskip

{\bf Matériel}~: 6750\texteuro.
\begin{itemize}[fullwidth]
    \item {\em Livres}~: 750\texteuro.
    Il s'agit de livres spécialisés au coût parfois élevé~: nous
    fixons le prix moyen à 75\texteuro, le montant correspond donc
    à une dizaine de livres. L'idée est de constituer une petite
    \og bibliothèque \fg{} accessible à tous les membres du projet.

    \item {\em Ordinateurs}~: 6000\texteuro.
    Il s'agit de renouveler du matériel vieillissant.
    L'expérimentation par le calcul formel est cruciale dans ce
    projet et nécessite des ordinateurs avec des capacités de calcul
    raisonnables. Nous prévoyons une machine par membre du projet à
    1500\texteuro{} chacune, soit 6000\texteuro{} (par exemple~:
    deux portables et deux fixes).

    \item {\em Autres.}
    Le budget non dépensé sur les deux points précédents servira
    éventuellement à du petit matériel: tablettes, clés USB, {\em etc.}
\end{itemize}
\smallskip
    
{\bf Missions~:} 7250\texteuro.
\begin{itemize}[fullwidth]
    \item {\em Conférences internationales}~: 5550\texteuro.
    Cette partie du budget aidera les participants à participer aux
    conférences du domaine (FPSAC, Séminaire Lotharingien de
    Combinatoire, {\em etc.}). Avec 1387,50\texteuro{} par participant,
    il s'agit d'un complément à d'autres sources de financement.

    \item {\em Rencontres et séances de travail}~: 1500\texteuro.
    Elles sont l'essence du projet. C'est le moment où nous
    chercherons des solutions aux problèmes posés, ferons le point,
    prendrons des décisions, {\em etc.} Nous prévoyons quatre
    rencontres d'une journée chacune, à raison d'une rencontre par
    trimestre. La plupart des participants étant à l'UPEM, les frais
    sont limités pour des rencontres en région parisienne. Le budget
    de 375\texteuro{} par rencontre servira à inviter des chercheurs
    qui travaillent sur des thèmes similaires et qui ne sont pas
    membres du projet, comme Frédéric Chapoton ou Loïc Foissy.
\end{itemize}

%%%%%%%%%%%%%%%%%%%%%%%%%%%%%%%%%%%%%%%%%%%%%%%%%%%%%%%%%%%%%%%%%%%%%%%%
%%%%%%%%%%%%%%%%%%%%%%%%%%%%%%%%%%%%%%%%%%%%%%%%%%%%%%%%%%%%%%%%%%%%%%%%
%%%%%%%%%%%%%%%%%%%%%%%%%%%%%%%%%%%%%%%%%%%%%%%%%%%%%%%%%%%%%%%%%%%%%%%%
\bibliographystyle{plain}
\bibliography{Projet}
\bigskip

\Large{\bf{Avis et visa du directeur d’unité}}
\bigskip

\end{document}
