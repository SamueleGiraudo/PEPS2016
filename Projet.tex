% Création : fév. 2015.
% Révision : Déc. 2015. pour appel 2016
\documentclass[10pt,reqno]{amsart}

\usepackage[utf8x]{inputenc}
\usepackage[francais]{babel}
\usepackage{amsmath,amsfonts,amssymb,amsthm,shuffle}
\usepackage[T1]{fontenc}
\usepackage{lmodern}

% Layout.
\usepackage[top=1.6cm,bottom=1.6cm,left=1.5cm,right=1.5cm]{geometry}

% Colors of hyperlinks.
\usepackage[dvipsnames]{xcolor}
\usepackage[hyperindex=true,frenchlinks=true,colorlinks=true,
citecolor=RoyalBlue,linkcolor=LimeGreen,urlcolor=Tan,linktocpage,
pagebackref=false]{hyperref}

% Tikz.
\usepackage{tikz}

% Misc.
\usepackage{cite}
\usepackage{subcaption}
\usepackage{ifthen}
\usepackage{textcomp} % symbole euro
\usepackage{calc} % encadrement de texte
\usepackage{array} % fixer la taille des colonnes

%%%%%%%%%%%%%%%%%%%%%%%%%%%%%%%%%%%%%%%%%%%%%%%%%%%%%%%%%%%%%%%%%%%%%%%%
%%%%%%%%%%%%%%%%%%%%%%%%%%%%%%%%%%%%%%%%%%%%%%%%%%%%%%%%%%%%%%%%%%%%%%%%
%%%%%%%%%%%%%%%%%%%%%%%%%%%%%%%%%%%%%%%%%%%%%%%%%%%%%%%%%%%%%%%%%%%%%%%%
\title{Projet PEPS \\
Dénombrement de prographes}
\date{\today}
\keywords{Combinatoire~; dénombrement~; opérades~; PROs~;
séries génératrices~; arbres.}

\author{Nicolas Borie}
\author{Jean-Paul Bultel}
\author{Samuele Giraudo}
\author{Matthieu Josuat-Vergès}

%%%%%%%%%%%%%%%%%%%%%%%%%%%%%%%%%%%%%%%%%%%%%%%%%%%%%%%%%%%%%%%%%%%%%%%%
%%%%%%%%%%%%%%%%%%%%%%%%%%%%%%%%%%%%%%%%%%%%%%%%%%%%%%%%%%%%%%%%%%%%%%%%
%%%%%%%%%%%%%%%%%%%%%%%%%%%%%%%%%%%%%%%%%%%%%%%%%%%%%%%%%%%%%%%%%%%%%%%%
\numberwithin{equation}{subsection}
\setcounter{tocdepth}{2}
\renewcommand{\leq}{\leqslant}
\renewcommand{\geq}{\geqslant}

%%%%%%%%%%%%%%%%%%%%%%%%%%%%%%%%%%%%%%%%%%%%%%%%%%%%%%%%%%%%%%%%%%%%%%%%
%%%%%%%%%%%%%%%%%%%%%%%%%%%%%%%%%%%%%%%%%%%%%%%%%%%%%%%%%%%%%%%%%%%%%%%%
%%%%%%%%%%%%%%%%%%%%%%%%%%%%%%%%%%%%%%%%%%%%%%%%%%%%%%%%%%%%%%%%%%%%%%%%
\tikzstyle{Noeud} = [circle,draw=Turquoise!80,fill=Turquoise!20,thick,
inner sep=0pt,minimum size=4mm,line width=1pt]
\tikzstyle{Arete}=[OrangeRed!80,cap=round,line width=1pt]
\tikzstyle{Feuille}=[rectangle,draw=Black!70,fill=Black!20,
inner sep=0pt,minimum size=1mm,line width=.75pt]
\tikzstyle{Operateur}=[rectangle,rounded corners,draw=YellowGreen!80,
fill=YellowGreen!20,minimum size=5mm,inner sep=1pt,line width=1pt]
\tikzstyle{Marque1}=[draw=YellowOrange!80,fill=YellowOrange!20]
\tikzstyle{Marque2}=[draw=SeaGreen!80,fill=SeaGreen!20]
\tikzstyle{Marque3}=[draw=Maroon!80,fill=Maroon!20]
\tikzstyle{Marque4}=[draw=Periwinkle!80,fill=Periwinkle!20]

%%%%%%%%%%%%%%%%%%%%%%%%%%%%%%%%%%%%%%%%%%%%%%%%%%%%%%%%%%%%%%%%%%%%%%%%
%%%%%%%%%%%%%%%%%%%%%%%%%%%%%%%%%%%%%%%%%%%%%%%%%%%%%%%%%%%%%%%%%%%%%%%%
%%%%%%%%%%%%%%%%%%%%%%%%%%%%%%%%%%%%%%%%%%%%%%%%%%%%%%%%%%%%%%%%%%%%%%%%
\newcommand{\Hilbert}{\mathcal{H}}
\newcommand{\Gen}{\mathcal{G}}
\newcommand{\OpLibre}{\mathrm{OL}}
\newcommand{\PROLibre}{\mathrm{PL}}
\newcommand{\Entrees}{\mathrm{ent}}
\newcommand{\Sorties}{\mathrm{sor}}
\newcommand{\Degre}{\mathrm{deg}}

\usepackage{ifthen}
\newboolean{draft}
\setboolean{draft}{true} % true affiche les todo et info, false les degagent
\ifdraft
\newcommand{\TODO}[2][To do: ]{\textcolor{red}{\textit{#1#2}}}
\newcommand{\INFO}[2][Info: ]{\textcolor{green}{\textit{#1#2}}}
\else
\newcommand{\TODO}[2][]{}
\newcommand{\INFO}[2][]{}
\fi

\newcommand{\Cacher}[1]{} % Commande qui efface son argument !

%%%%%%%%%%%%%%%%%%%%%%%%%%%%%%%%%%%%%%%%%%%%%%%%%%%%%%%%%%%%%%%%%%%%%%%%
%%%%%%%%%%%%%%%%%%%%%%%%%%%%%%%%%%%%%%%%%%%%%%%%%%%%%%%%%%%%%%%%%%%%%%%%
%%%%%%%%%%%%%%%%%%%%%%%%%%%%%%%%%%%%%%%%%%%%%%%%%%%%%%%%%%%%%%%%%%%%%%%%
\begin{document}

% \maketitle

\begin{tabular}{|>{\centering\arraybackslash}m{5.5cm}|>
    {\centering\arraybackslash}m{11.6cm}|} \hline
    \includegraphics{logo_CNRS.jpg} & \Large{APPEL A PROJETS} \\
    INS2I - 2016 & \Large{JCJC INS2I 2016} \\ \hline
\end{tabular}
\bigskip

\Large{\bf{Identification}}

\begin{tabular}{|m{5.5cm}|m{11.6cm}|} \hline
    Nom du porteur du projet & Samuele Giraudo (M.C.F. LIGM - UMR 8049)
        \\ \hline
    Adresse e-mail du porteur & {\tt samuele.giraudo@u-pem.fr} \\ \hline
    Titre long du projet & Pr{\bf o}priétés {\bf c}ombinatoires et
    {\bf a}lgébriques des {\bf P}ROs l{\bf i}bres\\ \hline
    Acronyme du projet & OCAPI \\ \hline
\end{tabular}
\bigskip

\Large{\bf{Résumé du projet}}

\fbox{\parbox{\linewidth-8\fboxrule-8\fboxsep}{Les prographes
    sont des assemblages plans de boites ayant plusieurs entrées et
    plusieurs sorties. Ces objets combinatoires généralisent un grand
    nombre de structures discrètes bien connues comme les arbres (une
    seule sortie pour chaque n\oe ud), les permutations ou encore les
    opérades. Toutefois, les prographes sont aujourd'hui peu
    connus. Notamment, comme il s'agit d'objets plans, déterminer la
    structure de données à utiliser en machine pour une génération 
    exhaustive et efficace est déjà un défi intéressant. Suivant le 
    statut considéré pour les boites, les prographes permettent de modéliser 
    des situations propres au calcul symbolique (les boites sont des 
    opérateurs algébriques), à la théorie des graphes et des flots (les 
    boites dont les n\oe uds d'un réseau) ou encore en électronique (les 
    boites sont des portes logiques). D'un point de vue algébriques,
    les prographes sont les éléments d'une structure algébrique connue 
    sous le nom de PRO libre. Le projet OCAPI propose une étude combinatoire 
    et algébrique des prographes.
    %% Les prographes
    %% sont des assemblages plans d'opérateurs ayant possiblement plusieurs
    %% entrées et plusieurs sorties. Ces objets combinatoires généralisent
    %% un grand nombre de structures discrètes bien connues. Par exemple,
    %% on peut voir les arbres binaires comme des prographes dans lesquels
    %% le seul opérateur de base apparaissant possède deux entrées et une
    %% seule sortie. Toutefois, les prographes sont aujourd'hui peu connus.
    %% Notamment, comme l'objet est plan, déterminer la structure de données
    %% à utiliser en machine pour leur génération exhaustive efficace est
    %% déjà un défi intéressant. Ce projet a vocation à s'appuyer sur
    %% l'investigation expérimentale, dans un premier temps, pour la
    %% génération et le dénombrement des prographes. Dans un deuxième temps,
    %% nous tenterons d'établir des résultats théoriques sur le comptage
    %% des prographes selon diverses caractéristiques pour mieux comprendre
    %% ces objets.
}}
\bigskip

\Large{\bf{Autres membres du projet}}

\begin{tabular}{|l|c|l|c|} \hline
    \bf{Nom partenaire} & \bf{Qualité/} & \bf{e-mail partenaire}
        & \bf{Unité de} \\
    & \bf{Titre} &  & \bf{recherche} \\ \hline
    Nicolas Borie & M.C.F. & {\tt nicolas.borie@u-pem.fr}
        & LIGM - UMR 8049 \\ \hline
    Jean-Paul Bultel & Docteur & {\tt jean-paul.bultel@univ-rouen.fr}
        & LITIS - EA 4108 \\ \hline
  Matthieu Josuat-Vergès \qquad & C.R.2 & {\tt matthieu.josuat-verges@u-pem.fr}
   \qquad
    & LIGM - UMR 8049 \qquad \\ \hline
\end{tabular}
\bigskip

\Large{\bf{Description du projet}}

%%%%%%%%%%%%%%%%%%%%%%%%%%%%%%%%%%%%%%%%%%%%%%%%%%%%%%%%%%%%%%%%%%%%%%%%
%%                  UN JOLI PRO STYLE CIRCUIT IMPRIMÉ                %%%
%%%%%%%%%%%%%%%%%%%%%%%%%%%%%%%%%%%%%%%%%%%%%%%%%%%%%%%%%%%%%%%%%%%%%%%%

\begin{figure}[ht]
  \centering
  \begin{tikzpicture}[yscale=0.6, xscale=0.8]
    \tikzstyle{Boite} = [very thick]
    \tikzstyle{Profils} = [thick]

    % Cadre autour du PROgraphe
    \draw (0, 1) -- (5, 1);
    \draw (0, 1) -- (0, 8);
    \draw (5, 1) -- (5, 8);
    \draw (0, 8) -- (5, 8);
    
    % Opérateurs boites du PROgraphe
    \draw[Boite] (3.5, 2) -- (3.5, 3) -- (4.5, 3) -- (4.5, 2) -- (3.5, 2);
    \draw[Boite] (0.5, 2) -- (0.5, 3) -- (2.5, 3) -- (2.5, 2) -- (0.5, 2);

    \draw[Boite] (0.5, 4) -- (0.5, 5) -- (3.5, 5) -- (3.5, 4) -- (0.5, 4);

    \draw[Boite] (2.5, 6) -- (2.5, 7) -- (4.5, 7) -- (4.5, 6) -- (2.5, 6);
    \draw[Boite] (0.5, 6) -- (0.5, 7) -- (1.5, 7) -- (1.5, 6) -- (0.5, 6);

    % Points de soudure des boites
    \fill[black] (1, 2) circle (0.065cm);
    \fill[black] (2, 2) circle (0.065cm);
    \fill[black] (1.5, 3) circle (0.065cm);
    \fill[black] (4, 3) circle (0.065cm);
    \fill[black] (2.5, 4) circle (0.065cm);
    \fill[black] (4, 2) circle (0.065cm);
    \fill[black] (1.5, 4) circle (0.065cm);
    \fill[black] (1.25, 5) circle (0.065cm);
    \fill[black] (2, 5) circle (0.065cm);
    \fill[black] (2.75, 5) circle (0.065cm);
    \fill[black] (4, 6) circle (0.065cm);
    \fill[black] (1, 6) circle (0.065cm);
    \fill[black] (3, 6) circle (0.065cm);
    \fill[black] (1, 7) circle (0.065cm);
    \fill[black] (3.5, 7) circle (0.065cm);

    % Fils du PROgraphe
    \draw[Profils] (1, 0.5) -- (1, 2);
    \draw[Profils] (2, 0.5) -- (2, 2);
    \draw[Profils] (3, 0.5) -- (3, 3.5) -- (2.5, 3.5) -- (2.5, 4);
    \draw[Profils] (4, 0.5) -- (4, 2);

    \draw[Profils] (1.5, 3) -- (1.5, 4);
    \draw[Profils] (4, 3) -- (4, 6);

    \draw[Profils] (1.25, 5) -- (1.25, 5.5) -- (1, 5.5) -- (1, 6);
    \draw[Profils] (2.75, 5) -- (2.75, 5.5) -- (3, 5.5) -- (3, 6);


    \draw[Profils] (1, 7) -- (1, 7.5) -- (1.25, 7.5) -- (1.25, 8.5);
    \draw[Profils] (2, 5) -- (2, 7.5) -- (2.5, 7.5) -- (2.5, 8.5);
    \draw[Profils] (3.5, 7) -- (3.5, 7.5) -- (3.75, 7.5) -- (3.75, 8.5);

    % Entrées du PROgraphe
    \fill[black] (1, 0.5) circle (0.15cm);
    \draw (1, 0.5) node[below] {$e_1$};
    \fill[black] (2, 0.5) circle (0.15cm);
    \draw (2, 0.5) node[below] {$e_2$};
    \fill[black] (3, 0.5) circle (0.15cm);
    \draw (3, 0.5) node[below] {$e_3$};
    \fill[black] (4, 0.5) circle (0.15cm);
    \draw (4, 0.5) node[below] {$e_4$};

    % Sorties du PROgraphe
    \fill[black] (1.25, 8.5) circle (0.15cm);
    \draw (1.25, 8.5) node[above] {$s_1$};
    \fill[black] (2.5, 8.5) circle (0.15cm);
    \draw (2.5, 8.5) node[above] {$s_2$};
    \fill[black] (3.75, 8.5) circle (0.15cm);
    \draw (3.75, 8.5) node[above] {$s_3$};
  \end{tikzpicture}
    
    \caption{Un prographe constitué de quatre entrées
     $e_1$, $e_2$, $e_3$ et $e_4$ et de trois sorties $s_1$, $s_2$ et 
     $s_3$. Il contient deux boites atomiques ayant une entrée et une 
     sortie, deux boites atomiques ayant deux entrées et une sortie et 
     une boite atomique ayant deux entrées et trois sorties.}
    \label{fig:exemple_prographe_circuit}
\end{figure}

%%%%%%%%%%%%%%%%%%%%%%%%%%%%%%%%%%%%%%%%%%%%%%%%%%%%%%%%%%%%%%%%%%%%%%%%
%%%%%%%%%%%%%%%%%%%%%%%%%%%%%%%%%%%%%%%%%%%%%%%%%%%%%%%%%%%%%%%%%%%%%%%%
%%%%%%%%%%%%%%%%%%%%%%%%%%%%%%%%%%%%%%%%%%%%%%%%%%%%%%%%%%%%%%%%%%%%%%%%


\section{Contexte}

%%%%%%%%%%%%%%%%%%%%%%%%%%%%%%%%%%%%%%%%%%%%%%%%%%%%%%%%%%%%%%%%%%%%%%%%
%%%%%%%%%%%%%%%%%%%%%%%%%%%%%%%%%%%%%%%%%%%%%%%%%%%%%%%%%%%%%%%%%%%%%%%%
\subsection{Les opérades}
L'algèbre universelle est une branche des mathématiques et de
l'informatique dont le rôle est de répertorier et d'étudier les
structures algébriques. Il en existe de plusieurs sortes, des monoïdes
aux algèbres de Lie, en passant par les treillis et les algèbres de Hopf.
Le point commun à ces objets est qu'ils sont constitués d'un ensemble
---~ou bien d'un espace vectoriel~--- sur lequel agissent des opérateurs.
Ces opérateurs vérifient un certain nombre de relations~; par exemple,
le produit des monoïdes est associatif et l'opération des algèbres de
Lie ---~le crochet de Lie~--- est antisymétrique et vérifie l'identité
de Jacobi.

Toutes ces classes de structures algébriques, qui sont de ce fait
décrites par leurs opérations et les relations qu'elles vérifient,
interviennent en informatique et en combinatoire de manière cruciale. À
titre d'exemple, le monoïde libre constitue le socle de base de la
théorie des langages et un très grand nombre d'exemples récents (voir
par exemple~\cite{HNT05,Gir12}) montre que les algèbres de Hopf
combinatoires apportent des propriétés très précises sur les objets
combinatoires qu'elles mettent en jeu. L'étude d'un point de vue général
de toutes ces classes de structures algébriques possède ainsi des
avantages très clairs.

L'un des outils modernes les plus adéquats et les plus prometteurs pour
obtenir des résultats dans ce sens est offert par la théorie des
{\em opérades}, dont les premières impulsions proviennent de la
topologie algébrique~\cite{May72}. D'un point de vue intuitif, une opérade
est un espace vectoriel d'opérateurs que l'on peut composer entre eux
(plus de détails sur les opérades sont répertoriés dans~\cite{Mar08,LV12}).
Les opérades changent ainsi le référentiel de l'étude~: nous ne considérons
plus, comme c'est le cas dans une structure algébrique, des produits
d'éléments via des opérateurs mais manipulons plutôt des opérateurs et
les composons pour en obtenir des nouveaux. Ce changement de point de
vue offre plusieurs avantages. Parmi eux, nous avons le fait qu'une
opérade encode à elle seule toute une catégorie de structures algébriques.
Par exemple, l'{\em opérade associative} représente tous les monoïdes et
l'{\em opérade de Lie} représente toutes les algèbres de Lie. La
compréhension d'une opérade et de ses propriétés donne beaucoup
d'informations sur la totalité des éléments de la classe de structures
algébriques qu'elle encode.

En outre, même si elles proviennent historiquement d'un contexte purement
algébrique, les opérades occupent aujourd'hui une place de choix en
combinatoire. En effet, la définition d'opérades sur des objets
combinatoires permet d'en extraire des propriétés, d'obtenir des
résultats énumératifs ou encore d'établir des bijections entre différentes
familles d'objets combinatoires~\cite{Cha08,CG14,Gir15}. De plus, l'étude
des opérades fait appel à des outils combinatoires classiques, comme les
séries génératrices et la combinatoire de diverses familles d'arbres.

%%%%%%%%%%%%%%%%%%%%%%%%%%%%%%%%%%%%%%%%%%%%%%%%%%%%%%%%%%%%%%%%%%%%%%%%
%%%%%%%%%%%%%%%%%%%%%%%%%%%%%%%%%%%%%%%%%%%%%%%%%%%%%%%%%%%%%%%%%%%%%%%%
\subsection{Les PROs}
La théorie des opérades possède cependant quelques inconvénients,
principalement dus à ses limitations. Tout d'abord, les opérades ne
permettent pas de décrire les groupes, structures algébriques pourtant
omniprésentes. Elle n'offrent pas non plus la possibilité de décrire des
structures algébriques ayant une opération distributive sur une autre,
comme c'est le cas dans les semi-anneaux. Une autre limitation, orthogonale
à ces dernières, réside dans le fait que les opérades permettent de
modéliser seulement des structures algébriques dans lesquelles les
opérateurs disposent d'exactement une sortie. Les algèbres de Hopf, qui
disposent d'un coproduit, une opération à une entrée et deux sorties, ne
peuvent ainsi pas être modélisées par les opérades.

Cette dernière limitation est levée en s'autorisant à enrichir l'espace
des opérateurs propres aux opérades avec des opérateurs à plusieurs
sorties. Ces objets, les {\em PROs}, introduits par Mac Lane~\cite{McL65}
(de manière surprenante historiquement avant les opérades), généralisent
les opérades et permettent de décrire les algèbres de Hopf ainsi que
d'autres structures algébriques encore plus complexes. Un PRO,
abréviation de {\em product category}, est ainsi un ensemble (ou un
espace vectoriel) d'opérateurs à un nombre arbitraire d'entrées et de
sorties, que l'on peut composer entre eux.

L'étude générale des PROs mène, comme c'était le cas pour l'étude des
opérades, à l'obtention de propriétés sur des structures algébriques,
cette fois-ci plus complexes. Un angle d'attaque pour obtenir des
résultats dans ce sens consiste à concentrer initialement l'étude sur
les {\em PROS libres}~\cite{Laf03,Mar08,BG14}. Ces structures sont
définies comme suit. Étant donné un ensemble bigradué de générateurs
\begin{equation}
    \Gen :=
    \bigsqcup_{p \geq 1} \bigsqcup_{q \geq 1} \Gen(p, q),
\end{equation}
nous appelons {\em prographe élémentaire sur $\Gen$} tout élément de
$\Gen$. Un élément de $\Gen(p, q)$ peut se voir comme un opérateur
à $p$ entrées et $q$ sorties (et se dessine comme tel). Un
{\em prographe sur $\Gen$} est un opérateur qui peut être obtenu en
composant des prographes élémentaires sur $\Gen$ entre eux
(voir la figure~\ref{fig:exemple_prographe}).
\begin{figure}[ht]
    \centering
    \begin{equation*}\footnotesize
        \begin{split}
        \begin{tikzpicture}[xscale=.24,yscale=.19]
            \node[Feuille](S1)at(2,0){};
            \node[Feuille](S2)at(4,0){};
            \node[Feuille](S3)at(5,0){};
            \node[Feuille](S4)at(7,0){};
            \node[Feuille](S5)at(9,0){};
            \node[Feuille](S6)at(13,0){};
            \node[Feuille](S7)at(16,0){};
            \node[Feuille](S8)at(18,0){};
            \node[Feuille](S9)at(19,0){};
            %
            \node[Feuille](E1)at(2,-10){};
            \node[Feuille](E2)at(4,-10){};
            \node[Feuille](E3)at(5,-10){};
            \node[Feuille](E4)at(6,-10){};
            \node[Feuille](E5)at(8,-10){};
            \node[Feuille](E6)at(10,-10){};
            \node[Feuille](E7)at(12,-10){};
            \node[Feuille](E8)at(14,-10){};
            \node[Feuille](E9)at(15,-10){};
            \node[Feuille](EA)at(16,-10){};
            \node[Feuille](EB)at(17,-10){};
            \node[Feuille](EC)at(18,-10){};
            \node[Feuille](ED)at(19,-10){};
            %
            \node[Operateur](N1)at(3,-5){\begin{math}{\tt a}\end{math}};
            \node[Operateur](N2)at(7,-8){\begin{math}{\tt a}\end{math}};
            \node[Operateur,Marque3](N3)at(9,-2){\begin{math}{\tt b}\end{math}};
            \node[Operateur](N4)at(11,-8){\begin{math}{\tt a}\end{math}};
            \node[Operateur](N5)at(13,-5){\begin{math}{\tt a}\end{math}};
            \node[Operateur,Marque2](N6)at(16,-5){\begin{math}{\tt c}\end{math}};
            %
            \draw[Arete](N1)--(S1);
            \draw[Arete](N1)--(S2);
            \draw[Arete](E3)--(S3);
            \draw[Arete](N2)--(S4);
            \draw[Arete](N3)--(S5);
            \draw[Arete](N5)--(S6);
            \draw[Arete](N6)--(S7);
            \draw[Arete](EC)--(S8);
            \draw[Arete](ED)--(S9);
            %
            \draw[Arete](N1)--(E1);
            \draw[Arete](N1)--(E2);
            \draw[Arete](N2)--(E4);
            \draw[Arete](N2)--(E5);
            \draw[Arete](N4)--(E6);
            \draw[Arete](N4)--(E7);
            \draw[Arete](N5)--(E8);
            \draw[Arete](N6)--(E9);
            \draw[Arete](N6)--(EA);
            \draw[Arete](N6)--(EB);
            %
            \draw[Arete](N2)--(N3);
            \draw[Arete](N3)--(N4);
            \draw[Arete](N3)--(N5);
            \draw[Arete](N4)--(N5);
        \end{tikzpicture}
        \end{split}
    \end{equation*}
    \caption{Un prographe sur $\Gen := \Gen(2, 2) \sqcup \Gen(3, 1)$
    où $\Gen(2, 2) := \{{\tt a}\}$ et $\Gen(3, 1) := \{{\tt b}, {\tt c}\}$.
    Il est de
    degré $6$, d'arité entrante $13$ et d'arité de sortie $9$.}
    \label{fig:exemple_prographe}
\end{figure}
Le {\em PRO libre sur $\Gen$} est l'ensemble $\PROLibre(\Gen)$ des
prographes sur $\Gen$ muni de l'opération permettant de les composer.

%%%%%%%%%%%%%%%%%%%%%%%%%%%%%%%%%%%%%%%%%%%%%%%%%%%%%%%%%%%%%%%%%%%%%%%%
%%%%%%%%%%%%%%%%%%%%%%%%%%%%%%%%%%%%%%%%%%%%%%%%%%%%%%%%%%%%%%%%%%%%%%%%
%%%%%%%%%%%%%%%%%%%%%%%%%%%%%%%%%%%%%%%%%%%%%%%%%%%%%%%%%%%%%%%%%%%%%%%%
\section{Objectifs}
Si $x$ est un prographe sur $\Gen$, le {\em degré} $\Degre(x)$ de $x$ est
le nombre de prographes élémentaires qui le constituent.
L'{\em arité entrante} $\Entrees(x)$ (resp. {\em de sortie}
$\Sorties(x)$) de $x$ est le nombre total d'entrées (resp. de sorties)
de $x$ (voir l'exemple en figure~\ref{fig:exemple_prographe}). La
{\em série de Hilbert} du PRO libre sur $\Gen$ est la série génératrice
trivariée
\begin{equation} \label{equ:serie_hilbert_PROs}
    \Hilbert_{\Gen}(y, z, t)
    := \sum_{x \in \PROLibre(\Gen)}
    y^{\Entrees(x)} \; z^{\Sorties(x)} \; t^{\Degre(x)}.
\end{equation}
Celle-ci dénombre les prographes sur $\Gen$ selon leur degré et leurs
arités d'entrée et de sortie.

Si le calcul des séries de Hilbert des opérades libres (série analogue
à~\eqref{equ:serie_hilbert_PROs} dans le contexte des opérades) est
quelque chose de bien connu, le calcul des séries de Hilbert des PROs
libres ne l'est pas. Le dénombrement des prographes selon leur degré et
leurs arités d'entrée et de sortie est ainsi une question ouverte à
laquelle nous souhaitons apporter des éléments de réponse. De ce fait,
obtenir une expression pour $\Hilbert_{\Gen}(y, z, t)$, étant donné
un ensemble de générateurs $\Gen$, est l'objectif majeur de ce projet.

Dans un travail récent, certains des membres de ce projet ont eu des
résultats dans ce sens~\cite{BG14} pour le dénombrement des prographes
libres sur un générateur ayant le même nombre d'entrées que de sorties.
Ce résultat s'appuie sur le {\em lemme d'inversion} de Viennot~\cite{Vie86},
introduit dans le contexte du dénombrement d'empilements de pièces. En
effet, un prographe élémentaire, qui possède un même nombre $\ell$
d'entrées que de sorties, peut être vu comme une pièce de largeur $\ell$
d'un empilement de pièces. Le cas général semble bien plus complexe à
traiter et semble nécessiter des outils combinatoires différents.

Nous espérons, à l'issue de ce projet, soumettre un article à un journal
de combinatoire algébrique ainsi qu'un article à une conférence du
même domaine.

%%%%%%%%%%%%%%%%%%%%%%%%%%%%%%%%%%%%%%%%%%%%%%%%%%%%%%%%%%%%%%%%%%%%%%%%
%%%%%%%%%%%%%%%%%%%%%%%%%%%%%%%%%%%%%%%%%%%%%%%%%%%%%%%%%%%%%%%%%%%%%%%%
%%%%%%%%%%%%%%%%%%%%%%%%%%%%%%%%%%%%%%%%%%%%%%%%%%%%%%%%%%%%%%%%%%%%%%%%

%% \section{Méthodologie}
%% L'une des questions préliminaires au problème soumis consiste à obtenir
%% un moyen de coder les prographes qui soit adapté à l'extraction
%% d'informations pour leur dénombrement. Une piste prometteuse pour cela
%% consiste à s'inspirer des outils de génération combinatoire déjà
%% disponibles dans le logiciel \texttt{Sage}~\cite{Sage} comme la construction
%% \texttt{SearchForest} développée par Nicolas Borie. Écrire un programme
%% qui accepte en entrée un ensemble de générateurs $\Gen$, et génère ainsi
%% tous les prographes sur $\Gen$ d'un degré fixé et ayant des arités
%% d'entrée et de sortie données, constituera un outil de base pour ce travail.

%% En effet, l'obtention de données énumératives par le biais
%% d'expérimentations informatiques nous semble être une stratégie efficace
%% pour regrouper des informations sur les prographes. Les suites d'entiers
%% qui dénombrent les prographes selon certaines de leurs caractéristiques,
%% obtenues par ce procédé, peuvent être recoupées avec les informations
%% de l'{\em Encyclopédie en ligne des suites de nombres entiers}~\cite{Slo}.
%% Cette approche permet, dans un premier temps, de conjecturer des formules
%% explicites pour l'énumération de prographes. De plus, si les suites
%% d'entiers obtenues par ces expérimentations vérifient des récurrences,
%% celles-ci nous offriront des pistes sur la bonne manière d'appréhender les
%% prographes en terme d'assemblages ou de désassemblages d'objets.

%%%%%%%%%%%%%%%%%%%%%%%%%%%%%%%%%%%%%%%%%%%%%%%%%%%%%%%%%%%%%%%%%%%%%%%%
%%%%%%%%%%%%%%%%%%%%%%%%%%%%%%%%%%%%%%%%%%%%%%%%%%%%%%%%%%%%%%%%%%%%%%%%
%%%%%%%%%%%%%%%%%%%%%%%%%%%%%%%%%%%%%%%%%%%%%%%%%%%%%%%%%%%%%%%%%%%%%%%%
\section{Perspectives}
La connaissance de la série $\Hilbert_{\Gen}(y, z, t)$ pour tout ensemble
de générateurs $\Gen$, ainsi que le processus nécessaire à sa découverte,
mènent à de multiples perspectives.

La première est de nature purement combinatoire. En effet, la connaissance
du nombre de prographes sur un ensemble de générateurs $\Gen$ permet
d'établir l'existence de bijections entre des familles d'objets
combinatoires et informatiques (comme des mots, des arbres, des graphes,
des cartes, {\em etc.}) et certains prographes. Ceci offrirait un moyen
de coder par des prographes des objets existants et donc de les
appréhender d'une façon inédite. On peut ainsi espérer obtenir de
nouvelles propriétés combinatoires et même algorithmiques sur les objets
ainsi codés.

Un deuxième axe est de nature algébrique. En effet, la connaissance de
$\Hilbert_{\Gen}(y, z, t)$ donne des informations sur les dimensions
des PROs libres, ce qui est une donnée algébrique fondamentale. Nous
pensons de plus que la compréhension des prographes, nécessaire à
leur dénombrement, va se baser sur des propriétés algébriques des
PROs libres qui restent à découvrir. Il semble, en effet, qu'il soit
nécessaire de poser une définition alternative de la catégorie des
PROs pour la rendre davantage adaptée à notre problème.

Un troisième axe mélange les deux aspects précédents. Depuis environ une
vingtaine d'années, la théorie des algèbres de Hopf combinatoires a connu
une forte expansion et on cherche toujours de nouveaux objets
combinatoires sur lesquels construire des algèbres de Hopf. Les PROs
étant des objets plus généraux et moins étudiés, la théorie n'est pas
aussi riche. Cependant, il est clair que certains objets combinatoires,
qui ne rentrent pas dans le cadre des algèbres de Hopf, seront tout à fait
abordables sous l'angle des PROs. L'objectif est donc de montrer que
c'est l'outil naturel pour étendre le spectre d'objets abordables par la
combinatoire algébrique.

Tous ces résultats potentiels peuvent se connecter avec des travaux
récents sur les PROs, soit étudiés d'un point de vue de la théorie du
calcul (voir~\cite{Laf03}), soit d'un point de vue algébrique
(voir~\cite{Val07}).

%%%%%%%%%%%%%%%%%%%%%%%%%%%%%%%%%%%%%%%%%%%%%%%%%%%%%%%%%%%%%%%%%%%%%%%%
%%%%%%%%%%%%%%%%%%%%%%%%%%%%%%%%%%%%%%%%%%%%%%%%%%%%%%%%%%%%%%%%%%%%%%%%
%%%%%%%%%%%%%%%%%%%%%%%%%%%%%%%%%%%%%%%%%%%%%%%%%%%%%%%%%%%%%%%%%%%%%%%%
\section{Membres du projet}
Nous décrivons ici la constitution de l'équipe du projet et expliquons
en quoi elle nous semble adaptée. Le domaine de recherche des
quatre membres s'inscrit en combinatoire algébrique~; chacun possède
néanmoins ses spécialités.

Le projet est porté par {\bf Samuele Giraudo}
(\url{http://igm.univ-mlv.fr/~giraudo/}). Il a effectué sa thèse à
l'université Paris-Est Marne-la-Vallée au LIGM sous la direction de
Jean-Christophe Novelli et Florent Hivert, soutenue le 8 décembre 2011.
Il occupe depuis 2012 un poste de maître de conférences en informatique
au LIGM dans l'équipe combinatoire algébrique et calcul symbolique. Son
domaine principal est la théorie des opérades. Il a notamment
obtenu dans ce contexte une construction générale d'opérades à partir de
monoïdes~\cite{Gir15} et un moyen de comprendre les propriétés d'une
opérade en la voyant comme une certaine opérade colorée~\cite{CG14}. Ce
membre a travaillé et obtenu des résultats sur les PROs~\cite{BG14}.
Étant donné que la plupart des notions connues sur les opérades peuvent
se généraliser aux PROs, ses connaissances sur les opérades pourront être
utiles dans ce travail.

{\bf Nicolas Borie} (\url{http://www-igm.univ-mlv.fr/~borie/}) a effectué
sa thèse à l'université Paris-Sud XI au LRI sous la direction de Nicolas
Thiéry, soutenue le 7 décembre 2011. Il occupe depuis 2013 un poste de
maître de conférences en informatique au LIGM dans l'équipe combinatoire
algébrique et calcul symbolique. Ce membre développe depuis de nombreuses
années pour \texttt{Sage}. Ses connaissances en programmation et en
expérimentation informatique seront nécessaires pour mener le projet à
bien. L'un de ses résultats~\cite{Bor13} consiste en la conception d'un
algorithme efficace de génération de mots d'entiers soumis à certaines
conditions, dont certaines idées semblent pouvoir être exploitées pour
la génération des prographes. Son domaine principal est la théorie
effective des invariants, domaine qu'il est le seul à avoir exploré
parmi les membres de ce groupe.

{\bf Jean-Paul Bultel} a effectué sa thèse à l'université Paris-Est
Marne-la-Vallée au LIGM sous la direction de Jean-Yves Thibon, soutenue
le 25 novembre 2011. Il est actuellement membre associé au LIGM et au
LITIS (Université de Rouen). Ce membre a déjà travaillé et obtenu des
résultats sur les PROs~\cite{BG14}. Son domaine principal est la théorie
des algèbres de Hopf combinatoires (voir~\cite{Bul11}). Ses connaissances
des structures algébriques sur des opérateurs à plusieurs entrées et
plusieurs sorties sera nécessaire au déroulement de ce travail.

{\bf Matthieu Josuat-Vergès} (\url{http://igm.univ-mlv.fr/~josuatv/})
a effectué sa thèse à l'université Paris-Sud XI au LRI sous la
direction de Sylvie Corteel, soutenue le 25 janvier 2010. Il occupe
depuis 2011 un poste de chargé de recherche CNRS au LIGM dans l'équipe
combinatoire algébrique et calcul symbolique. Ce membre possède une
vision très large et l'expérience la plus grande en combinatoire parmi
les membres de ce groupe. Il a par ailleurs récemment obtenu des
résultats énumératifs sur des familles d'arbres~\cite{JV15}. Ses
connaissances en combinatoire vont potentiellement pouvoir mettre en
évidence, au fil du traitement de ce projet, des connexions non
évidentes entre le sujet du projet et d'autres thèmes.

%%%%%%%%%%%%%%%%%%%%%%%%%%%%%%%%%%%%%%%%%%%%%%%%%%%%%%%%%%%%%%%%%%%%%%%%
%%%%%%%%%%%%%%%%%%%%%%%%%%%%%%%%%%%%%%%%%%%%%%%%%%%%%%%%%%%%%%%%%%%%%%%%
%%%%%%%%%%%%%%%%%%%%%%%%%%%%%%%%%%%%%%%%%%%%%%%%%%%%%%%%%%%%%%%%%%%%%%%%
\section{Budget}
Le budget demandé est 14000\texteuro. Voici sa répartition.

\begin{itemize}
    \item {\bf Matériel}~: 6750\texteuro.
    \begin{itemize}
        \item {\em Livres}~: 750\texteuro.
        Il s'agit de livres spécialisés au coût parfois élevé~: nous
        fixons le prix moyen à 75\texteuro, le montant correspond donc
        à une dizaine de livres. L'idée est de constituer une petite
        \og bibliothèque \fg{} accessible à tous les membres du projet.

        \item {\em Ordinateurs}~: 6000\texteuro.
        Il s'agit de renouveler du matériel vieillissant.
        L'expérimentation par le calcul formel est cruciale dans ce
        projet et nécessite des ordinateurs avec des capacités de calcul
        raisonnables. Nous prévoyons une machine par membre du projet à
        1500\texteuro{} chacune, soit 6000\texteuro{} (par exemple~:
        deux portables et deux fixes).

        \item {\em Autres.}
        Le budget non dépensé sur les deux points précédents servira
        éventuellement à du petit matériel: tablettes, clés USB, {\em etc.}
    \end{itemize}

    \item {\bf Missions~:} 7250\texteuro.
    \begin{itemize}
        \item {\em Conférences internationales}~: 5550\texteuro.
        Cette partie du budget aidera les participants à participer aux
        conférences du domaine (FPSAC, Séminaire Lotharingien de
        Combinatoire, {\em etc.}). Avec 1387,50\texteuro{} par participant,
        il s'agit d'un complément à d'autres sources de financement.

        \item {\em Rencontres et séances de travail}~: 1500\texteuro.
        Elles sont l'essence du projet. C'est le moment où nous
        chercherons des solutions aux problèmes posés, ferons le point,
        prendrons des décisions, {\em etc.} Nous prévoyons quatre
        rencontres d'une journée chacune, à raison d'une rencontre par
        trimestre. La plupart des participants étant à l'UPEM, les frais
        sont limités pour des rencontres en région parisienne. Le budget
        de 375\texteuro{} par rencontre servira à inviter des chercheurs
        qui travaillent sur des thèmes similaires et qui ne sont pas
        membres du projet, comme Frédéric Chapoton ou Loïc Foissy.
    \end{itemize}
\end{itemize}

%%%%%%%%%%%%%%%%%%%%%%%%%%%%%%%%%%%%%%%%%%%%%%%%%%%%%%%%%%%%%%%%%%%%%%%%
%%%%%%%%%%%%%%%%%%%%%%%%%%%%%%%%%%%%%%%%%%%%%%%%%%%%%%%%%%%%%%%%%%%%%%%%
%%%%%%%%%%%%%%%%%%%%%%%%%%%%%%%%%%%%%%%%%%%%%%%%%%%%%%%%%%%%%%%%%%%%%%%%
\bibliographystyle{plain}
\bibliography{Projet}
\bigskip

\Large{\bf{Avis et visa du directeur d’unité}}
\bigskip

\end{document}
